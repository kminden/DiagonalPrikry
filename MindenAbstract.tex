\documentclass{amsart}
\newcommand{\CH}{\textup{\textsf{CH}}}
\begin{document}
Kaethe Minden

The subcompleteness of diagonal Prikry forcing
\section*{Abstract}
Subcomplete forcing is a novel class of forcing notions, originally defined by Ronald Bj\"orn Jensen around 2009. Until quite recently, Jensen's writings made up the vast majority of the literature on the subject. Indeed, the definition of subcompleteness in and of itself is daunting. In my research, I attempt to make the subject somewhat more approachable to set theorists, and have shown various preservation properties of subcomplete forcing that one might desire of a forcing class. 

Countably closed and proper forcing notions are well known in the field of set theory and have been around for some time; there is a wealth of literature on what sort of preservation properties they enjoy. While all countably closed forcing notions are also subcomplete, as they are equivalent to complete forcing notions, there are well-known and widely used forcing notions which are subcomplete but not proper. In general, subcomplete forcing is a class of forcing notions that do not add reals, and have similar preservations properties to those of countably closed forcing. Unlike proper forcing, they may potentially alter cofinalities to $\omega$, whereas proper forcings satisfy a countable covering property and thus cannot do so. Indeed, no $ccc$ forcing is subcomplete.
Examples of subcomplete forcing include Prikry forcing, and Namba forcing (under $\CH$), as shown by Jensen~\cite[Section 3.3]{Jensen:2012fr}. 
In a similar vein, Fuchs~\cite{Fuchs:2017Magidor} has shown that Magidor forcing is subcomplete. 

In this talk I plan to give an introduction to subcomplete forcing, and the methodology of showing a forcing notion is subcomplete, which involves Barwise theory. Specifically I plan to give an idea as to how to show that a simplified version of what I refer to as generalized diagonal Prikry forcing is subcomplete, which hints to us that many Prikry-like forcings are subcomplete. In particular, letting $D$ be an infinite discrete set of measurable cardinals, I show that the forcing to add a point below each cardinal in $D$ is subcomplete. 

\bibliographystyle{abbrv}
\bibliography{./ApplicationStuff/CV-RS-Pubs/BIB}

\end{document}
