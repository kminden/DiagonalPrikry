%%%%%%%%%%%%%%%%%%%%%%%%%%%%%%
%ASL 2019
%%%%%%%%%%%%%%%%%%%%%%%%%%%%%%

\documentclass[utf8x,xcolor=svgnames,8pt]{beamer}

\usepackage{tikz} \usetikzlibrary{matrix} \usetikzlibrary{patterns}
\usepackage{bbm}

\usepackage{appendixnumberbeamer}
% Add total frame count to slides, optional. From Stefan,
% http://www.latex-community.org/forum/viewtopic.php?f=4&t=2173
%\expandafter\def\expandafter\insertshorttitle\expandafter{%
 % \insertshorttitle\hfill\insertframenumber\,/\,\inserttotalframenumber}


%\addtolength{\textwidth}{1.5em}
%\addtolength{\oddsidemargin}{-0.75em}

%\renewcommand{\labelenumi}{\textbf{\arabic{enumi}.}}
%\renewcommand{\baselinestretch}{1.2}
\renewcommand{\bar}{\overline}
%\renewcommand{\labelenumi}{\arabic{enumi}.}%\renewcommand{\labelenumi}{\textbf{\arabic{enumi}.}}
%\renewcommand{\baselinestretch}{1.2}

%\setlist[description]{leftmargin=5.5em,labelindent=\parindent}

\renewcommand{\P}{\mathbb{P}}
\newcommand{\Q}{\mathbb{Q}}
\newcommand{\M}{\mathcal{M}}
\newcommand{\Namba}{\mathbb{N}}
\newcommand{\D}{\mathbb{D}}
\newcommand{\N}{{\overline{N}}}
\renewcommand{\H}{\overline{H}}
\renewcommand{\G}{\overline{G}}
\renewcommand{\S}{{\overline{S}}}
\newcommand{\R}{\mathbb{R}}
%\newcommand{\C}{\pmb{\mathcal C}}
%\newcommand{\C}{\leftexp{<\omega}{2}}
%\newcommand{\B}{\mathbb{B}_\mathcal{N}}
\newcommand{\B}{\mathbb{B}}
\renewcommand{\U}{\mathcal{U}}
\renewcommand{\c}{\mathfrak{c}}

\newcommand{\PFA}{\textup{\ensuremath{\textsf{PFA}}}}
\newcommand{\MA}{\textup{\ensuremath{\textsf{MA}}}}
\newcommand{\MM}{\textup{\ensuremath{\textsf{MM}}}}
\newcommand{\SPFA}{\textup{\ensuremath{\textsf{SPFA}}}}
\newcommand{\ZFC}{\textup{\ensuremath{\textsf{ZFC}}}}
\newcommand{\SCFA}{\textup{\ensuremath{\textsf{SCFA}}}}
\newcommand{\BSCFA}{\textup{\ensuremath{\textsf{BSCFA}}}}
\newcommand{\BPFA}{\textup{\ensuremath{\textsf{BPFA}}}}
\newcommand{\CH}{\textup{\textsf{CH}}}
\newcommand{\GCH}{\textup{\ensuremath{\textsf{GCH}}}}
\newcommand{\Ord}{\textup{\ensuremath{\text{Ord}}}}
\newcommand{\id}{\textup{\ensuremath{\text{id}}}}
\newcommand{\MP}{\textup{\ensuremath{\textsf{MP}}}}
\newcommand{\bfMP}{\textup{\ensuremath{\textsf{\textbf{MP}}}}}
\newcommand{\RA}{\textup{\ensuremath{\textsf{RA}}}}
\newcommand{\bfRA}{\textup{\ensuremath{\textsf{\textbf{RA}}}}}
\newcommand{\BFA}{\textup{\ensuremath{\textsf{BFA}}}}
\newcommand{\HC}{\textup{\ensuremath{H_{\omega_1}}}}

\DeclareMathOperator{\cof}{cof}
\DeclareMathOperator{\height}{height}
\DeclareMathOperator{\ran}{range}
\DeclareMathOperator{\otp}{otp}
\DeclareMathOperator{\cp}{cp}
\DeclareMathOperator{\dom}{dom}
\DeclareMathOperator{\Succ}{succ}
\DeclareMathOperator{\wfc}{wfc}
\DeclareMathOperator{\BA}{BA}
\DeclareMathOperator{\Add}{\mathcal A\textit{dd}\,}
\DeclareMathOperator{\Coll}{\mathcal C\textit{oll}\,}
\DeclareMathOperator*{\bigdoublevee}{\bigvee\mkern-15mu\bigvee}

\newcommand{\To}{\longrightarrow}
\newcommand{\st}{\; | \;}
\newcommand{\set}[2]{\left\{#1\st #2 \right\}}
\newcommand{\seq}[2]{\langle #1 \st #2 \rangle}

\newcommand{\Ptail}{\P_{\mathrm{tail}}}
\newcommand{\Gtail}{G_{\mathrm{tail}}}

\newcommand{\leftexp}[2]{{\vphantom{#2}}^{#1}{#2}}
\newcommand{\funcs}[2]{\leftexp{#1}{#2}}

\newcommand{\forces}{\Vdash}
\newcommand{\proves}{\vdash}
\newcommand{\rest}{\mathbin{\upharpoonright}}
\newcommand{\meet}{\wedge}
\newcommand{\Meet}{\bigwedge}

\DeclareMathOperator{\MPsc}{ \textup{\textsf{MP}}_{\textit{sc}}}
\newcommand{\bfMPsc}{\textbf{\textup{\textsf{MP}}}_{\textit{sc}}}
\DeclareMathOperator{\MPc}{\textup{\textsf{MP}}_{\textit{c}}}
\newcommand{\bfMPc}{\textbf{\textup{\textsf{MP}}}_{\textit{c}}}
\DeclareMathOperator{\MPp}{\textup{\textsf{MP}}_{\textit{p}}}
\newcommand{\bfMPp}{\textbf{\textup{\textsf{MP}}}_{\textit{p}}}
\DeclareMathOperator{\MPccc}{\textup{\textsf{MP}}_{\textit{ccc}}}
\newcommand{\bfMPccc}{\textbf{\textup{\textsf{MP}}}_{\textit{ccc}}}
\DeclareMathOperator{\MPsp}{ \textup{\textsf{MP}}_{\textit{sp}}}
\newcommand{\bfMPsp}{\textbf{\textup{\textsf{MP}}}_{\textit{sp}}}
\DeclareMathOperator{\MPsemi}{ \textup{\textsf{MP}}_{\textit{semi}}}
\newcommand{\bfMPsemi}{\textbf{\textup{\textsf{MP}}}_{\textit{semi}}}

\newcommand{\bflMPsc}{ \textup{\textsf{\textbf{LMP}}}_{\textit{sc}} }
\newcommand{\bflMPc}{ \textup{\textsf{\textbf{LMP}}}_{\textit{c}} }
\newcommand{\bflMPccc}{ \textup{\textsf{\textbf{LMP}}}_{\textit{ccc}} }
\newcommand{\bflMPp}{ \textup{\textsf{\textbf{LMP}}}_{\textit{p}} }
\newcommand{\bflMPsemi}{ \textup{\textsf{\textbf{LMP}}}_{\textit{semi}} }
\newcommand{\bflMPsp}{ \textup{\textsf{\textbf{LMP}}}_{\textit{sp}} }

\newcommand{\lMP}[2]{ \textup{\textsf{MP}}^{H_{#2}}_{#1} (H_{#2}) }

\newcommand{\bflMP}{ \textup{\textsf{\textbf{LMP}}} }



\DeclareMathOperator{\RAsc}{\textup{\textsf{RA}}_{\textit{sc}}}
\DeclareMathOperator{\RAc}{\textup{\textsf{RA}}_{\textit{c}}}
\DeclareMathOperator{\RAp}{\textup{\textsf{RA}}_{\textit{p}}}
\DeclareMathOperator{\RAccc}{\textup{\textsf{RA}}_{\textit{ccc}}}
\DeclareMathOperator{\RAsp}{\textup{\textsf{RA}}_{\textit{sp}}}
\DeclareMathOperator{\RAsemi}{\textup{\textsf{RA}}_{\textit{semi}}}
\DeclareMathOperator{\bfRAsc}{\textup{\textsf{\textbf{RA}}}_{\textit{sc}}}
\DeclareMathOperator{\bfRAc}{\textup{\textsf{\textbf{RA}}}_{\textit{c}}}
\DeclareMathOperator{\bfRAccc}{\textup{\textsf{\textbf{RA}}}_{\textit{ccc}}}
\DeclareMathOperator{\bfRAp}{\textup{\textsf{\textbf{RA}}}_{\textit{p}}}
\DeclareMathOperator{\bfRAsp}{\textup{\textsf{\textbf{RA}}}_{\textit{sp}}}
\DeclareMathOperator{\bfRAsemi}{\textup{\textsf{\textbf{RA}}}_{\textit{semi}}}


\newcommand{\SH}{\mathcal{S}\textit{k} \,}
\newcommand{\sk}[3]{\SH^{#1}( {#2} \cup {\ran(#3)} ) }
\newcommand{\Sk}[3]{\SH^{#1}( {#2} \cup {#3} ) }

\newcommand{\TC}[1]{\mathrm{TC}(\{ #1 \})}


\mode<presentation>
\setbeamertemplate{navigation symbols}{}
\usetheme{Boadilla}
\usecolortheme[named={ForestGreen}]{structure}

\makeatother
\setbeamertemplate{footline}
{
  \leavevmode%
  \hbox{%
  \begin{beamercolorbox}[wd=.25\paperwidth,ht=2.25ex,dp=1ex,center]{author in head/foot}%
    \usebeamerfont{author in head/foot}\insertshortauthor
  \end{beamercolorbox}%
  \begin{beamercolorbox}[wd=.45\paperwidth,ht=2.25ex,dp=1ex,center]{title in head/foot}%
    \usebeamerfont{title in head/foot}\insertshorttitle
  \end{beamercolorbox}%
  \begin{beamercolorbox}[wd=.3\paperwidth,ht=2.25ex,dp=1ex,center]{date in head/foot}%
  	\usebeamerfont{date in head/foot}\insertshortdate\hspace*{3em}
    \insertframenumber{} / \inserttotalframenumber%\hspace*{1ex}
  \end{beamercolorbox}}%
  \vskip0pt%
}
\makeatletter

%\useoutertheme{split}

%\renewcommand{\emph}{\textit}

\newtheorem{proposition}{Proposition}
\newtheorem{question}{Question}

\setbeamertemplate{enumerate items}[default]

\title{The subcompleteness of diagonal Prikry forcing}
\author{Kaethe Minden}
\institute[Marlboro]{\includegraphics[scale=0.16]{mc-logo.png} \\ Vermont, USA}
\date[ASL 2019]{Association for Symbolic Logic 2019 North American Annual Meeting \\ 20 May 2019}

\begin{document}
\begin{frame}[plain]
\titlepage
\end{frame}

\section{Diagonal Prikry Forcing}
\subsection{Generalized Diagonal Prikry Forcing is Subcomplete}

\section{Introduction}
\begin{frame}
\frametitle{Subcomplete forcing} 
Subcomplete forcings do not add reals.
%but may potentially alter cofinalities to $\omega$. %Defined at some point before 2012.
\begin{block}{Examples of subcomplete forcing}
\begin{itemize}
	\item \textbf{Countably closed forcing}.
	\item \textit{\textbf{Namba forcing}}, which adds an $\omega$-cofinal sequence to $\omega_2$, under $\CH$ \cite{Jensen:2014}.
	\item \textbf{Prikry forcing}, which forces a measurable cardinal to have cofinality $\omega$ while preserving cardinalities \cite{Jensen:2014}.
	\item \textbf{Magidor forcing}, which collapses the cofinality of a measurable carrying a Mitchell-increasing $\omega_1$-sequence of normal ultrafilters to $\omega_1$ \cite{Fuchs:2017Magidor}.
%	\item Certain kinds of \textbf{generalized diagonal Prikry forcing} which force a sequence of measurables to have cofinality $\omega$.
%	\item The forcing $\mathbb P_{\textbf{A}}$, which shoots a closed set of order type $\omega_1$ through a stationary set $A \subseteq \kappa$ of $\omega$-cofinal ordinals, and is used to force $\neg \Box_\kappa$.
	\item \color{Green} \textbf{Generalized Prikry forcing} as defined by Fuchs (\cite{Fuchs:2005kx}).
\end{itemize}
\end{block}
%\begin{itemize}
	Revised countable support ($rcs$) iterations of subcomplete forcings are subcomplete.
	%\item Lottery sums of subcomplete forcings are subcomplete.
	%\item If $\P$ is subcomplete and $\pi: \P \To \Q$ is a dense embedding, then $\Q$ is subcomplete. 
%\end{itemize}
\end{frame}

\begin{frame}
How subcompleteness fits in with other forcing classes which preserve stationary subsets of $\omega_1$:
\begin{center}
\begin{figure}[h!]
    \begin{tikzpicture}
      \draw (0:0cm) circle (3.4cm);
      \node at (90:2.5cm) {Subproper};
      \draw (90:0cm) circle (1cm) node [text=black] { Complete};
      \node at (90:-0.4cm){ ($\sigma$-cl.)};
      \node at (80:-1.4cm){?};
      \draw (180:1cm) circle (2cm);
      \node at (180:2cm) { Proper};
      \draw (330:1cm) circle (2cm);
      \node at (340:2cm){ Subcomplete};
       \draw[color=ForestGreen] (150:1.8cm) circle (.4cm) node [text=ForestGreen] {\footnotesize $ccc$};
    \end{tikzpicture}
\end{figure} 
\end{center}
\end{frame}

%\begin{frame}{The Standard Setup}
%Suppose that $\P$ is a forcing notion and $s$ is a set, with $\P, s \in H_\theta$, where $\theta$ is sufficiently large. Instead of just working with $H_\theta$ and its well order, the models used for defining subcompleteness will have the form $L_\tau[A] \supseteq H_\theta$ where $L_\tau[A] \models \ZFC^-$ such that $\tau>\theta$ is a (possibly singular) cardinal, and $A \subseteq \tau$. Write $N$ is a \emph{\textbf{suitable model}} for $\P,s$ above $\theta$ for this, with $N=L_\tau[A]$.
%\end{frame}
\begin{frame}{Definition of Completeness}
In order to define subcomplete forcing, we first define the notion of complete forcing. 

\vspace{1em}

Suppose that $\P$ is a forcing notion and $s$ is a set, with $\P, s \in H_\theta$, where $\theta$ is sufficiently large. Instead of just working with $H_\theta$ and its well order, the models used for defining completeness and subcompleteness have the form $L_\tau[A] \supseteq H_\theta$ where $L_\tau[A] \models \ZFC^-$ with $\tau>\theta$. Say $N$ is a \emph{\textbf{suitable model}} for $\P,s$ above $\theta$ if $N=L_\tau[A]$ in such a situation.
%which turns out to be equivalent to countably closed forcings. The definiton of subcomplete forcing is a tweaking of this definition.
%Roughly speaking, complete forcings posit that below some condition, generics for countable substructures extend to generics for the larger structure.
\begin{definition}
A forcing notion $\P$ is \emph{\textbf{complete}} so long as, for any set $s$ and for sufficiently large $\theta$ we have that whenever $N$ is a suitable model for $\P,s$ above $\theta$ and: \begin{itemize}
	\item $\sigma: \N \cong X \prec N$ where $X$ is countable and $\N$ is transitive;
	\item $\sigma(\overline \theta, \overline{\P}, \overline s)=\theta, \P, s$;
\end{itemize}
then, if $\overline G$ is $\overline{\P}$-generic over $\N$, there is a condition $p \in \P$ such that whenever $G \ni p$ is $\P$-generic, 
\begin{itemize}
	\item $\sigma ``\, \G \subseteq G$. 
\end{itemize}

In particular, this means $p$ forces that $\sigma$ lifts to an elementary embedding $\sigma^*:\overline N[\G] \To N[G].$
\end{definition}

%\frametitle{Complete forcing}
\begin{proposition}[Jensen] \begin{itemize}
	\item If $\P$ is countably closed then $\P$ is complete.
%\begin{proof} Let $\mathbb P$ be countably closed, and suppose we have $\theta >> |\mathbb P|$ satisfying the standard setup: \begin{itemize}
%	\item $\mathbb P \in H_\theta \subseteq N = L_\tau[A] \models \textsf{ZFC}^-, \tau>\theta$
%	\item $\sigma: \overline N \cong X \prec N$ where $X$ is countable and $\overline N$ is transitive
%\end{itemize}
%and let $\overline G \subseteq \overline{\mathbb P}$ be $\overline N$-generic. We can find such a $\overline G$ since $\overline N$ is countable. Take $p \leq \bigcup \sigma``\overline G$, which works since $\mathbb P$ is in fact countably directed closed.
%\end{proof}
%\end{frame}
%
%\begin{frame}
%\frametitle{Complete forcing}
	\item If $\P$ is complete, then $\P$ is forcing equivalent to a countably closed poset. 
	\end{itemize}
	\end{proposition}
%\begin{proof}[Proof sketch]
%Let $\theta$ verify the completeness of $\mathbb P$. Define a poset $\mathbb Q$ of conditions $q = \langle X_q, G_q \rangle$ such that $\mathbb P, \theta \in X_q \prec N$ where $X_q$ is countable and $G_q$ is $\mathbb P$--generic over $X_q$. Let $$q \leq r \iff X_q \supseteq X_r \text{ and } G_r = G_q \cap X_r.$$ Clearly $\mathbb Q$ is countably closed. Define  $\pi: \mathbb Q \rightarrow \text{BA}(\mathbb P)$ by taking $\pi(q) = \cap G_q$. That $\cap G_q \neq \emptyset$ follows as $\mathbb P$ is complete. Also verify: \begin{itemize}
%	\item $q \leq r \implies \cap G_q \leq \cap G_r$ %If $q \leq r$ then $G_r \subseteq G_q$, so the desired result follows.
%	\item $q || r \iff \cap G_q \bigcap \cap G_r \neq \emptyset$  %If $s \leq q, r$ then $s \in G_q \cap G_r$. For the other direction, let $\cap G_q \bigcap \cap G_r \neq \emptyset$. Then let $X \prec H_\theta$ with $X$ countable, $X_q \cup X_r \subseteq X$, and $\cap G_q \bigcap \cap G_r \in X$. Since $X$ is countable we may obtain $G \subseteq \mathbb P$ $X$-generic, with $\cap G_q \bigcap \cap G_r \in G$. Then $\langle X,G \rangle \leq q,r$.
%	\item $D=\{ \cap G_q \;|\; q \in \mathbb Q \}$ is dense in $\mathbb P$ %Let $p \in \mathbb P$. Let $X \prec H_\theta$ with $X$ countable and $p \in X$. Then if $G$ is $X$-generic and contains $p$, $p \leq \cap G$ and $\cap G \in D$.
%\end{itemize}
%\end{proof}
%\end{frame}
%
%\subsection{Subcomplete Forcing}
%\begin{frame}
%\frametitle{Definition of subcompleteness}
%\begin{block}{}
%We now define subcomplete forcing. Generally speaking, instead of requiring the original embedding to lift we only ask that below some condition, there is an embedding sufficiently similar to the original one which lifts.
%\end{block}
\end{frame}




\begin{frame}{Definition of Subcompleteness}
\begin{definition}
A forcing notion $\P$ is \emph{\textbf{subcomplete}} so long as, for any set $s$ and 
for sufficiently large $\theta$ we have that whenever we are in \emph{the standard setup} i.e. $N$ is a suitable model for $\P,s$ above $\theta$ and \begin{itemize}
	\item $\sigma: \N \cong X \prec N$ where $X$ is countable and $\N$ is full; $\textcolor{ForestGreen}{\bigstar}$
	\item $\sigma(\overline \theta, \overline{\P}, \overline s)=\theta, \P, s$;
\end{itemize}
then we have that if $\G$ is  $\overline{\P}$-generic over $\N$ then there is a condition $p \in \P$ such that whenever $G \ni p$ is $\P$-generic, there is $\sigma' \in V[G]$ satisfying: \begin{enumerate}
	\item $\sigma': \N \To N$;
	\item $\sigma'(\overline \theta, \overline{\P}, \overline s)=\theta, \P, s$;
	\item \label{item:skolemcompatibility} $\sk{N}{|\P|}{\sigma'} = \Sk{N}{|\P|}{X}$; $\textcolor{ForestGreen}{\bigstar}$
	\item  \label{item:sigmaprimelifts}$\sigma'``\, \G \subseteq G$.
\end{enumerate}
Thus $p$ forces that there is an embedding $\sigma'$ in the extension $V[G]$ which lifts to an embedding $\sigma'^*:\N[\G] \To N[G]$ in $V[G]$.
%We say that such a $\theta$ as above \textit{verifies the subcompleteness} of $\P$.
%Often we write $\delta$ instead of $\delta(\P)$ for the weight of $\P$ when there should be no confusion as to which poset $\P$ we are working with.%\footnote{See Section \ref{subsec:delta} for more on $\delta(\P)$.}
\end{definition}
\end{frame}

\begin{frame}
\frametitle{Generalized Diagonal Prikry Forcing}
\begin{definition}
Let $D$ be an infinite discrete set of measurable cardinals. For $\kappa \in D$ let $\U=\seq{U(\kappa)}{\kappa \in D}$ be a sequence of normal measures.

Define $\D=\D(\U)$, \emph{\textbf{generalized diagonal Prikry forcing}} from $\U$, by taking conditions of the form  
$( s, A )$ satisfying the following:
\begin{itemize}
	\item The \textit{stem}, $s$, is a function with domain $[D]^{<\omega}$ taking each measurable cardinal $\kappa \in \dom(s)$ to some ordinal $s(\kappa) < \kappa$.
	\item The \textit{upper part}, $A$, is a function with domain $D \setminus {\dom(s)}$ taking each $\kappa \in \dom(A)$ to some measure-one set $A(\kappa) \in U(\kappa)$.
\end{itemize}
Let $( s, A ) \leq ( t, B )$ so long as $s \supseteq t$, for all $\kappa \in \dom(s) \setminus \dom(t)$, $s(\kappa) \in B(\kappa)$ and for all $\kappa \in \dom(A)$, $A(\kappa) \subseteq B(\kappa)$.
If $G$ is a generic filter for $\D$, then its associated $\D$-generic sequence is \[S = S_G = \bigcup \set{ s }{ \exists A \ ( s, A ) \in G }.\qedhere\]
\end{definition}
\begin{block}{Genericity Criterion (Fuchs \cite{Fuchs:2005kx})} \label{fact:diagprikrymathias} Let $D$ be an infinite discrete set of measurable cardinals, with $\U$ a corresponding list of measures $\langle U(\kappa) \;|\; \kappa \in D \rangle$. Then an increasing sequence of ordinals $S = \langle S(\kappa) \;|\; \kappa \in D \rangle$ is a $\D(\U)$-generic  sequence if and only if for all $\mathcal X = \langle X_\kappa \in U(\kappa) \;|\; \kappa \in D \rangle$, the set $\{ \kappa \in D \;|\; S(\kappa) \notin X_\kappa\} \text{ is finite.}$
\end{block}
\end{frame}

\begin{frame}
\frametitle{Generalized Diagonal Prikry Forcing is Subcomplete}
\begin{theorem} Let $D$ be an infinite discrete set of measurable cardinals. %, such that $\otp(D)$ is less than the first limit point of $D$. 
Let $\U = \seq{ U(\kappa) }{ \kappa \in D }$ be a list of measures associated to $D$. Then $\D=\D(\U)$ is subcomplete. \end{theorem}

\textbf{Proof sketch:}

\vspace{0.5em}

In order to show that $\D$ is subcomplete, let $\theta$ be large enough and let $c$ be a set, and suppose we are in the standard setup, so $N$ is a suitable model for $\D, \U, c$ above $\theta$  and: \begin{itemize}
	\item $\sigma: \N \cong X \prec N$ where $X$ is countable and $\N$ is full
	\item $\sigma(\overline \theta, \overline{\D}, \overline{\U}, \overline c)=\theta, \D, \U, c$.
\end{itemize}
Let $\S \subseteq \overline{\D}$ be generic over $\N$. 

We need to argue that there is a $\D$-generic sequence $S$ and $\sigma' \in V[S]$ such that: \begin{itemize}
	\item $\sigma': \N \To N$ elementarily
	\item $\sigma'(\overline \theta, \overline{\D}, \overline{\U}, \overline c)=\theta, \D, \U, c$
	\item $\sk{N}{|\D|}{\sigma'} = \Sk{N}{|\D|}{X}$	
	\item $\sigma' `` \S \subseteq S$
\end{itemize}
The proof amounts to showing that certain infinitary theories, $\mathcal L$ and $\mathcal L'$, positing the existence of such a generic sequence $S$ and such an embedding $\sigma'$, are consistent.
\end{frame}

%\begin{frame}
%\begin{figure}[h!] 
%\begin{tikzpicture}
%\draw (-1,0)--(-1,3);
%\node [below] at (-1,0) {$\N$};
%\node at (-1,2.95) {$\frown$};
%
%\node [left] at (-1,1) {$\overline \kappa(0) \in \overline D$};
%\node at (-1,1) {-};
%\node [left] at (-1,2) {$\overline \rho$};
%\node at (-1,2) {-};
%\node [left] at (-1,2.5) {$\overline \nu$};
%\node at (-1,2.5) {-};
%
%\draw [->] (-.8,1)--(.8,2);
%\draw [->] (-.8,2)--(.8,3);
%\draw [->] (-.8,2.5)--(.8,3.5);
%
%\node [right] at (1,2) {$\kappa(0) \in D$};
%\node at (1,2) {-};
%\node [right] at (1,3) {$\rho$};
%\node at (1,3) {-};
%\node [right] at (1,3.5) {$\nu$};
%\node at (1,3.5) {-};
%
%\draw (1,0)--(1,4);
%\node [below] at (1,0) {$N$};
%\node at (1,3.95) {$\frown$};
%
%\draw [->, thick] (-.7,-0.3)--(.7,-0.3);
%\node [above] at (0,-0.3) {$\sigma$};
%\end{tikzpicture}
%\caption{Diagram depicting $\sigma$.}\label{figure:N} 
%\end{figure}
%\end{frame}

\begin{frame}{Barwise Theory}
\begin{definition} A transitive structure $\M$ is \textbf{admissible} if it models the axioms of \textsf{Kripke-Platek Set Theory} (\textsf{KP}) which consists of the axioms of \textsf{Empty Set}, \textsf{Pairing}, \textsf{Union}, $\Sigma_0$-\textsf{Collection}, and $\Sigma_0$-\textsf{Separation}. \end{definition}

Barwise introduced the syntax and model theory for infinitary, but ``$M$-finitary," languages on admissible structures. The basic idea is to perform standard first order logic, replacing ``finite" (as in finite sentences and formulas and proofs) with ``in the admissible structure $\mathcal M$."

%If $\mathcal L$ is a $\Sigma_1(\M)$-definable language or theory, the rough idea is that to check whether a sentence is in $\mathcal L$, one should imagine enumerating the formulae of $\mathcal L$ to find a sentence and a witness to it in the structure $\M$.
%
\begin{definition}[$\in$-theory] Let $\M$ be admissible. An infinitary axiomatized theory $\mathcal L=\mathcal L(\M)$, with a fixed predicate $\dot \in$ and \textbf{special constants} denoted $\underline x$ for elements $x \in \M$, is called an \textbf{$\in$-theory} over $\M$. The underlying axioms for these $\in$-theories will always involve $\ZFC^-$ and some \textbf{Basic Axioms} ensuring that $\dot \in$ behaves nicely. \end{definition}

\begin{definition} Let $M$ be a transitive structure with potentially infinitely many predicates. A theory is $\Sigma_1(M)$ if the theory is $\Sigma_1$-definable, with parameters from $M$. \end{definition}
%In the above definition, it should be clarified that it is possible to consider the same $\in$-theory defined over different admissible structures; if $\M, \M'$ are both admissible, then we can consider both $\mathcal L(\M)$ and $\mathcal L(\M')$. The distinction is only as to where the special constants come from. 

Barwise Completeness essentially says that consistent $\Sigma_1(\M)$ $\in$-theories satisfying $\textsf{ZF}^-$ over countable admissible structures $\M$ have models.
\end{frame}


\begin{frame} \frametitle{Transferring consistency}

Recall that we have an elementary embedding $\sigma:\N \To N$. 

Let $\alpha > \omega$ be a regular cardinal in $\N$. 

\vspace{1em}

%\begin{definition}
By a \textbf{transitive liftup} of $\sigma$ up to $\alpha$ we mean a pair $\langle N_* , \sigma_* \rangle$ such that 
\begin{itemize} 
	\item $N_*$ is transitive;
	\item $\sigma_*:\N \To N_*$ is an \textbf{$\alpha$-cofinal embedding}, meaning elements of $N^*$ are of the form $\sigma_*(f)(\beta)$ for some $f: \gamma \To \N$ where $\beta < \sigma(\gamma)$ and $\gamma<\alpha$;
	\item $\sigma_* \upharpoonright H_{\alpha}^{\N}= \sigma \upharpoonright H_{\alpha}^{\N}$. 
\end{itemize}
%\end{definition}

\begin{block}{Interpolation (Jensen)}
\begin{enumerate}
	\item The transitive liftup $\langle N_*, \sigma_* \rangle$ of $\sigma$ up to $\alpha$ exists.
	\item There is a unique factor embedding $k_*:N_*  \To N$ such that $k_* \circ \sigma_* = \sigma$ and $k_* \rest \bigcup \sigma `` H^{\N}_\alpha = \id$.
\end{enumerate}
\end{block} 

\begin{block}{Transfer (Jensen)}
Suppose we are in the following situation:
\begin{enumerate}
	\item $N_1$ and $N_0$ are both liftups of $\sigma$ with factor embedding $k: N_1 \To N_0$;
	\item We have a $\in$-theory $\mathcal L$ defined over the least-rank admissible structures over $N_0$ and $N_1$ (say $\mathcal M_0$ and $\mathcal M_1$ respectively);
	\item $\mathcal L(\M_1)$ is $\Sigma_1(\langle N_1, \vec{p}\rangle)$ and $\mathcal L(\M_0)$ is $\Sigma_1(\langle N_0, k(\vec p) \rangle)$.
\end{enumerate}
Then, 
if $\mathcal L(\mathcal M_1)$ is consistent, it follows that $\mathcal L(\mathcal M_0)$ is also consistent. 
\end{block} 

\end{frame}

\begin{frame}
\begin{enumerate} 
	\item Let $\rho = |\D|$. Let $k_0 : N_0 \cong \Sk{N}{\rho}{X}$ where $N_0$ is transitive, and set $\sigma_0 = k_0^{-1} \circ \sigma$.  It turns out that $\langle N_0, \sigma_0\rangle$ is a transitive liftup of $\sigma$ up to $\overline \nu$ where $\nu = \rho^+$.
\end{enumerate}	
%	\pause
\begin{figure}[h!]
\begin{tikzpicture}
\draw (-3,0)--(-3,3);
\node [below] at (-3,0) {$\N$};
\node at (-3,2.95) {$\frown$};

\node [left] at (-3,0.7) {$\overline \kappa(0) \in \overline D$};
\node at (-3,0.7) {-};
\node [left] at (-3,2) {$\overline \rho$};
\node at (-3,2) {-};
\node [left] at (-3,2.5) {$\overline \nu$};
\node at (-3,2.5) {-};

\draw [->] (-2.8,0.7)--(0.8,2.2);
\draw [->] (-2.8,2)--(0.8,3.6);
%\draw [->] (-2.8,2.5)--(0.8,4.5);

\node at (1,2.2) {-};

\node at (1,3.6) {-};
\node [above left] at (1,4.2) {$\nu_0$};
\node at (1,4.2) {$\bullet$};

\draw (1,0) -- (1,5.5);
\node [below] at (1,0) {$N_0$};
\node at (1,5.45) {$\frown$};

\draw [->] (1.2,2.2)--(2.8,2.2);
\draw [->] (1.2,3.6)--(2.8,3.6);
\draw [->] (-2.8,2.5)--(2.8,5);

\node [right] at (3,2.2) {$\kappa(0) \in D$};
\node at (3,2.2) {-};
\node [right] at (3,3.6) {$\rho$};
\node at (3,3.6) {-};
\node [right] at (3,5) {$\nu$};
\node at (3,5) {-};

\draw (3,0)--(3,6);
\node [below] at (3,0) {$N$};
\node at (3,5.95) {$\frown$};

\draw [->, thick] (-2.7,-0.3)--(0.7,-0.3);
\draw [->, thick] (1.3, -0.3)--(2.7,-0.3);
\node [above] at (-1,-0.3) {$\sigma_0$};
\node [above] at (2,-0.3) {$k_0$};
\draw[->,thick] (-2.75,-0.4) to [out = -30, in =-150] node[above]{$\sigma$} (2.9, -0.5);
\end{tikzpicture}
 %\caption{Diagram depicting the liftup $\langle N_0, \sigma_0\rangle$.} \label{figure:N0N}
\end{figure}

\end{frame}

\begin{frame}
\begin{enumerate} \setcounter{enumi}{1}
	\item By Interpolation, we have a liftup $\langle N_1, \sigma_1\rangle$ of $\sigma$ up to $\overline \kappa(0)$. 
\end{enumerate}
\begin{figure}[h!] 
\begin{tikzpicture}
\draw (-3,0)--(-3,3);
\node [below] at (-3,0) {$\N$};
\node at (-3,2.95) {$\frown$};

\node [left] at (-3,0.4) {$\overline \kappa(0) \in \overline D$};
\node at (-3,0.4) {-};
\node [left] at (-3,2) {$\overline \rho$};
\node at (-3,2) {-};
\node [left] at (-3,2.5) {$\overline \nu$};
\node at (-3,2.5) {-};

\draw [->] (-2.8,0.4)--(1.3,2);

\draw (0,0)--(0,5.3);
\node [below] at (0,0) {$N_1$};
\node at (0,5.25) {$\frown$};
\node at (0,1.5) {$\bullet$};
\node [above left] at (0,1.5) {$\kappa_1(0) \in D_1$};

\draw [->] (-2.8,2)--(1.3,3.8);

\draw (1.5,0) -- (1.5,5.7);
\node [below] at (1.5,0) {$N_0$};
\node at (1.5,5.65) {$\frown$};
\node at (1.5,2) {-};
\node at (1.5,3.8) {-};
\node [above left] at (1.5,4.4) {$\nu_0$};
\node at (1.5,4.4) {$\bullet$};

\draw [->] (1.7,2)--(2.8,2);
\draw [->] (1.7,3.8)--(2.8,3.8);

\draw [->] (-2.8,2.5)--(2.8,5);

\node [right] at (3,2) {$\kappa(0) \in D$};
\node at (3,2) {-};
\node [right] at (3,3.8) {$\rho$};
\node at (3,3.8) {-};
\node [right] at (3,5) {$\nu$};
\node at (3,5) {-};

\draw (3,0)--(3,6);
\node [below] at (3,0) {$N$};
\node at (3,5.95) {$\frown$};

\draw [->, thick] (-2.7,-0.3)--(-0.3,-0.3);
\draw [->, thick] (0.3, -0.3)--(1.2,-0.3);
\draw [->, thick] (1.8, -0.3)--(2.7,-0.3);
\draw[->, thick] (-2.8,-0.4) to  [out=-30, in=-145] node[above]{$\sigma_0$} (1.3,-0.5);
\draw[->,thick] (-2.9,-0.5) to [out = -50, in =-140] node[above]{$\sigma$} (2.9, -0.5);

\node [above] at (-1.5,-0.3) {$\sigma_1$};
\node [above] at (0.75,-0.3) {$k$};
\node [above] at (2.25, -0.3) {$k_0$};
\end{tikzpicture}
%\label{figure:N1N0N} \caption{Diagram depicting the liftups $\langle N_0, \sigma_0\rangle$ and $\langle N_1, \sigma_1\rangle$.}
\end{figure} 
\end{frame}

\begin{frame}
\begin{enumerate} \setcounter{enumi}{4}
	\item Define the infinitary $\in$-theory $\mathcal L(\mathcal M_*)$ as follows: \begin{description}
	\item[predicates] $\in$ 
	\item[constants] $\mathring{\sigma}, \mathring S, \underline x$ for $x \in \mathcal M_*$
	%\item[parameters] $N_*$ and $\theta_*, \mu_*, \D_*, \U_*, \lambda_*, D_*, c_* \in N_*$
	\item[axioms] \begin{itemize} \item $\ZFC^-$ and \textbf{Basic Axioms}
		\item $\mathring \sigma : \underline \N \To \underline{N_*}$ is a $\underline{\overline \kappa(0)}$--cofinal embedding
		\item $\mathring{\sigma}(\overline{\underline{\theta}}, \overline{\underline{\D}}, \overline{\underline{\U}}, \overline{\underline c})=\underline{\theta_*}, \underline{\D_*}, \underline{\U_*}, \underline{c_*}$
		\item $\mathring S$ is a $\underline{\D_*}$-generic sequence over $\underline{N_*}$
		\item $\mathring \sigma ``\overline{\underline S} \subseteq \mathring S$.
	\end{itemize}
\end{description} 
	\item $\mathcal L(\M_1)$ is consistent: To see this, use the genericity criterion and the $\overline{\kappa}(0)$-cofinality to see that $\sigma_1``\overline S$ may be used to find a $\D_1$-generic sequence over $N_1$. Namely, first obtain a diagonal Prikry sequence $S_1' \subseteq \D_1$ generic over $N_1$. Define, in $V[S_1']$, a new sequence $S_1 \subseteq \D_1$ generic over $N_1$ as follows:
$$S_1(\kappa) = \begin{cases} S_1'(\kappa) &\text{ if } \kappa \in D_1 \setminus \sigma_1``\overline D \\
					\sigma_1(\overline S(\overline \kappa)) &\text{ if } \kappa = \sigma_1(\overline \kappa). \end{cases}$$ 
	\item Using Transfer, we then have that $\mathcal L(\M_0)$ is consistent.						
	\end{enumerate}
\end{frame}


\begin{frame}
%If a $\Sigma_1$ $\in$-theory over an admissible structure $\M$ is consistent then it is still consistent in a forcing extension collapsing $\M$ to be countable - since the theory is $\M$-finitary.
\begin{enumerate}\setcounter{enumi}{5}
	\item Work in $V[F]$, a generic extension collapsing the admissible structure $\mathcal M_0$ to be countable. By Barwise Completeness, this gives us a model of $\mathcal L(\M_0)$ which grants the existence of a diagonal Prikry sequence $S$ generic over $N_0$ and thus $V$, by the genericity criterion. We also get an embedding $\overset{*}\sigma: \N \To N$. This $\overset{*}\sigma$ is in $V[F]$ however, not in $V[S]\subseteq V[F]$.
	\item Let $\mathcal M$ be a large enough admissible structure in $V[S]$, define $\mathcal L'(\mathcal M)$ as follows: 
\begin{description}
	\item[predicates] $\in$ 
	\item[constants] $\dot{\sigma}, \underline x$ for $x \in \mathcal M$
	\item[axioms] \begin{itemize} \item $\ZFC^-$ and \textbf{Basic Axioms}.
		\item $\dot \sigma : \overline{\underline N} \To \underline N$ elementarily
		\item $\dot{\sigma}(\overline{\underline{\theta}}, \overline{\underline{\D}}, \overline{\underline{\U}}, \overline{\underline c})=\underline{\theta}, \underline{\D}, \underline{\U}, \underline{c}$
		\item $\sk{\underline N}{\underline \rho}{\dot{\sigma}} = \sk{\underline N}{\underline \rho}{\underline \sigma}$
		\item $\dot \sigma ``\overline{\underline S} \subseteq \underline S$.
	\end{itemize}
\end{description}
	\item By Barwise Correctness, $\mathcal L'(\mathcal M)$ is consistent, indeed $\overset{*}\sigma \in V[F]$ witnesses this. 
	\item Let $\pi: \tilde{\mathcal M} \To \mathcal M$ where $\tilde{\mathcal M}$ is countable and transitive. $\mathcal L'(\tilde{\mathcal \M})$ is consistent since inconsistencies could be pushed up via $\pi$. Thus by Barwise Completeness, we have a model, $\tilde{\mathfrak A}$, of $\mathcal L'(\tilde{\mathcal M})$.
	\item Let $\sigma' =\pi \circ \dot{\sigma}^{\tilde{\mathfrak A}}$. This works! \qedhere
\end{enumerate}
\end{frame}




\begin{frame}
\frametitle{Bibliography}
\bibliographystyle{amsalpha}
\bibliography{../ApplicationStuff/CV-RS-Pubs-Refs-extra/BIB}
\end{frame}

\begin{frame}
\center Thank you!
\end{frame}


\end{document}